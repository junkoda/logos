\documentclass[a4paper,11pt]{article}
\usepackage{amsmath}
\usepackage{amssymb}
\usepackage{fancyhdr}
\usepackage{hyperref}

% Use Times fonts
\usepackage{mathptmx}
\usepackage[scaled]{helvet}
\renewcommand{\ttdefault}{pcr}
\usepackage{bm}

\setlength{\oddsidemargin}{0pt}   %%% left margin
\setlength{\textwidth}{159.2mm}
\setlength{\topmargin}{0mm}
\setlength{\headheight}{10mm}
\setlength{\headsep}{10mm}         %%% length between header and test
\setlength{\textheight}{219.2mm}

\setlength{\parindent}{0pt}

\pagestyle{fancy}
\renewcommand{\headrulewidth}{0pt}
%\fancyhf{}
%\rhead{JK 0000 - \thepage}
%\lfoot{}
%\cfoot{}
\rfoot{\url{https://github.com/junkoda/logos}}

\begin{document}


\section{Formulae}
\subsection{1D power spectrum}

The 1D power spectrum of a field $\phi$ is defined by the Fourier transform of the two-point correlation function on a line,
\begin{equation}
  P_\phi^{1D}(k) = \int \! dr
                  \left\langle \phi(x, y, z) \phi(x', y, z) \right\rangle
                  e^{-ikx},
\end{equation}
where $r = x - x'$.

This become the projection of the 3D power spectrum,
\begin{align}
  P_\phi^{1D}(k) &= \int \! \frac{dk_y}{2\pi} \frac{dk_z}{2\pi}
                   P_\phi(k, k_y, k_z)\\
                &= \frac{1}{4\pi} \int_0^\infty d k_\perp^2 P_\phi(k, \bm{k}_\perp) 
\end{align}

The line-of-sight velocity power spectrum
\begin{equation}
  P_{vel} = \frac{f^2 \mu^2}{k^2} P(k)
\end{equation}

Velocity dispersion:
\begin{align}
  \sigma_u^2 \equiv \langle u(x)^2 \rangle
  &= \int \! \frac{d^3 k}{(2\pi)^3} P_{vel}(k)
   = \frac{f^2}{6\pi^2} \int_0^\infty P(k) dk \\
  &= \int \frac{dk}{2\pi} P^{1D}_{vel}(k) dk
\end{align}


%
% Theory
%
\section{Theory}

\subsection{Gaussian power Spectrum}

\begin{equation}
  P_{vel}(k) = A e^{-(k\sigma_P)^2}
\end{equation}

\begin{equation}
  A = 2\sqrt{\pi} \sigma_u^2 \sigma_P
\end{equation}

\begin{equation}
  P_{1D}(k) = \int_{-\infty}^\infty \! dr \, e^{-ikr} \left\{
  e^{-k^2 \sigma_u^2 \left[ 1 - \exp \left( -r^2/4 \sigma_P^2 \right) \right]}
  \right\}
\end{equation}
    


\label{LastPage}
\end{document}

%
% 
%
